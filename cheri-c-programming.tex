\documentclass[12pt,twoside,openright,a4paper]{article}
\setlength{\oddsidemargin}{-0.4mm} % 25 mm left margin
\setlength{\evensidemargin}{\oddsidemargin}
\setlength{\textwidth}{160mm}      % 25 mm right margin
\setlength{\topmargin}{-5.4mm}     % 20 mm top margin
\setlength{\headheight}{5mm}
\setlength{\headsep}{5mm}
\setlength{\footskip}{10mm}
\setlength{\textheight}{237mm}     % 20 mm bottom margin

%\documentclass[11pt]{article}
%\usepackage[UKenglish]{isodate}%UK date endian
\usepackage[headings]{fullpage}
\usepackage[hidelinks]{hyperref}

\usepackage{bytefield}
\usepackage{color}
\usepackage[scaled=0.8]{DejaVuSansMono}
\usepackage[T1]{fontenc}
\usepackage{listings}
\usepackage{subcaption}
\usepackage{times}
\usepackage{url}
\usepackage[svgnames]{xcolor}
\definecolor{lightgray}{gray}{0.8}
\usepackage{xspace}

\renewcommand{\UrlFont}{\ttfamily\small}

\lstset{basicstyle=\footnotesize\ttfamily}
%\newcommand{\ccode}[1]{\lstinline[language={C}]{#1}}
%\newcommand{\cxxcode}[1]{\lstinline[language={C++}]{#1}}
\newcommand{\ccode}[1]{{\small\ttfamily{#1}}}
\newcommand{\cxxcode}[1]{{\ccode{#1}}}
\newcommand{\cconst}[1]{{\ccode{#1}}}
\newcommand{\cfunc}[1]{{\ccode{#1()}}}
\newcommand{\cvar}[1]{{\ccode{#1}}}
\newcommand{\pathname}[1]{{\ccode{#1}}}
\newcommand{\commandline}[1]{{\ccode{#1}}}

\newcommand{\ptrdifft}{{\ccode{ptrdiff\_t}}\xspace}
\newcommand{\sizet}{{\ccode{size\_t}}\xspace}
\newcommand{\ssizet}{{\ccode{ssize\_t}}\xspace}
\newcommand{\vaddrt}{{\ccode{vaddr\_t}}\xspace}
\newcommand{\cuintptrt}{{\ccode{uintptr\_t}}\xspace}
\newcommand{\cintptrt}{{\ccode{intptr\_t}}\xspace}
\newcommand{\ccharstar}{{\ccode{char *}}\xspace}
\newcommand{\cvoidstar}{{\ccode{void *}}\xspace}
\newcommand{\clongt}{{\ccode{long}}\xspace}
\newcommand{\cintt}{{\ccode{int}}\xspace}
\newcommand{\cintttt}{{\ccode{int32\_t}}\xspace}
\newcommand{\cintsft}{{\ccode{int64\_t}}\xspace}

\newcommand{\SIGPROT}{{\ccode{SIGPROT}}\xspace}

\newcommand{\note}[2]{{\color{blue}[ Note: #1 - #2]}}
\renewcommand{\note}[2]{}
\newcommand{\arnote}[1]{\note{#1}{Alex R.}}
\newcommand{\bdnote}[1]{\note{#1}{Brooks D.}}
\newcommand{\rwnote}[1]{\note{#1}{Robert W.}}
\newcommand{\amnote}[1]{\note{#1}{Alfredo M.}}
\newcommand{\psnote}[1]{\note{#1}{Peter S.}}
\newcommand{\pgnnote}[1]{\note{#1}{Peter N.}}

\hyphenation{Free-BSD}
\hyphenation{Free-RTOS}
\hyphenation{Cheri-BSD}
\hyphenation{Cheri-Free-RTOS}
\hyphenation{Cheri-ABI}
\hyphenation{Web-Kit}

\title{CHERI Pure-Capability C/C++ Programming Guide \\ (DRAFT)}
\author{Robert N. M. Watson, Brooks Davis, Alexander Richardson, \\
  John Baldwin, Jessica Clarke, Nathaniel Filardo, Simon W. Moore, \\
  Edward Napierala, Peter Sewell, and Peter G. Neumann}

\begin{document}
\sloppy

%% CL tech-report format provides its own cover page.  Comment for final
%% version.
\maketitle

%% CL tech-report format requires page numbering to start at 3.  Uncomment for
%% final version.
%\setcounter{page}{3}
%%

%
% Keep Abstract in sync with the Introduction.
%
\begin{abstract}
We briefly introduce the pure-capability C and C++ programming languages.
We describe the most commonly encountered differences between those languages
on CHERI versus on conventional architectures, and where existing software may
require minor changes.
We explain how modest language extensions allow selected software, such
as memory allocators, to further refine permissions and bounds on pointers.
This guidance is based on our experience adapting the FreeBSD operating-system
userspace, and applications such as PostgreSQL and WebKit, to run in a
pure-capability programming environment.
We also recommend further reading, including our pertinent technical reports
and papers.
\end{abstract}

\newpage
\setcounter{tocdepth}{2}
\tableofcontents

\newpage

\section{Introduction}

%
% Keep Abstract in sync with the Introduction.
%
This document is a brief introduction to the pure-capability CHERI C/C++
programming languages.
It describes the most commonly encountered differences between those languages
on CHERI versus on conventional architectures, and where existing software may
require minor changes.
It also explains how modest language extensions allow selected software, such
as memory allocators, to further refine permissions and bounds on pointers.
This guidance is based on the experience of adapting the FreeBSD
operating-system userspace, and applications such as PostgreSQL and WebKit, to
run in a pure-capability programming environment.
Section~\ref{sec:further_reading} recommends further reading, including our
pertinent technical reports and papers.

\textbf{XXXRW: There will be a new section on added compiler warnings and
  errors, and what to do about them.}

\textbf{NOTE: We have included some notes on potential areas of future change
  as we continue to refine CHERI C/C++ and its toolchain.}

\section{Background}

CHERI extends conventional processor Instruction-Set Architectures (ISAs) with
support for \textit{architectural capabilities}.
One important use for this new hardware data type is in the implementation
of safer C/C++ pointers and the code or data they point at.
Our technical report, \textit{An Introduction to CHERI}, provides a high-level
overview of the CHERI architecture, ISA modeling, hardware implementations,
and software stack~\cite{UCAM-CL-TR-941}.

\subsection{CHERI capabilities}

\begin{figure}[b]
\hspace{2.5cm}
% Tag
\begin{subfigure}[t!]{0.1\textwidth}
\begin{bytefield}[bitwidth=3pt]{1}
% \bitheader[endianness=big]{~,~} \\
\begin{leftwordgroup}{1-bit tag}
\bitbox{1}{}
\end{leftwordgroup}
\end{bytefield}
\end{subfigure}
% Capability
\begin{subfigure}[t!]{0.1\textwidth}
\begin{bytefield}[bitwidth=3pt]{64}
\bitheader[endianness=big]{0,63} \\
\begin{rightwordgroup}{128-bit \\ in-memory \\ capability}
\bitbox{16}{perms} & \bitbox{3}{\color{lightgray}\rule{\width}{\height}} & \bitbox{15}{otype} & \bitbox{30}{bounds} \\
\bitbox[lrb]{64}{64-bit~address}
\end{rightwordgroup}
\end{bytefield}
\end{subfigure}
\caption{128-bit CHERI Concentrate capability representation: 64-bit address
  and metadata in addressable memory; and 1-bit out-of-band tag.}
\label{figure:cheri-capability-representation}
\end{figure}

CHERI capabilities are twice the width of the native integer pointer type of
the baseline architecture: there are 128-bit capabilities on 64-bit platforms,
and 64-bit capabilities on 32-bit platforms.
Each capability consists of an integer (virtual) address of the natural size for
the architecture (e.g., 32 or 64 bit), and also additional metadata that is
compressed in order to fit in the remaining 32 or 64 bits of the capability
(see Figure~\ref{figure:cheri-capability-representation}).
In addition, they are associated with a 1-bit validity ``tag'' whose value is
maintained in registers and memory by the architecture.
Each element of the capability contributes to the protection model:

\begin{description}
\item[Validity tag] The tag tracks the validity of a capability.
  If invalid, the capability cannot be used for load, store, instruction
  fetch, or other operations.
  It is still possible to extract fields from an invalid capability,
  including its address.
  Subject to precision limits of the bounds compression model, a capability's
  address (i.e., where it points) can be changed while maintaining the
  validity tag.

\item[Bounds] The lower and upper bounds are addresses restricting the
  portion of the address space within which the capability can be used for
  load, store, and instruction fetch.

\item[Permissions] The permissions mask controls how the capability can be
  used -- for example, by providing loading and storing of data and/or
  capabilities.

\item[Object type] If not equal to $-1$, the capability is ``sealed'' and
  cannot be modified or dereferenced, but can be used to implement opaque
  pointer types.
  This feature is not described further in this document, as it is primarily
  used to implement software compartmentalization rather than object-level
  memory protection.
\end{description}

When stored in memory, valid capabilities must be naturally aligned -- i.e., at
64-bit or 128-bit boundaries, depending on capability size -- as that is the
granularity at which in-memory tags are maintained.
Partial or complete overwrites with data, rather than a complete overwrite
with a valid capability, lead to the in-memory tag being cleared, preventing
corrupted capabilities from later being dereferenced.

In order to reduce the memory footprint of capabilities, capability
compression is used to reduce the overhead of bounds so that the full
capability, including address, permissions, and bounds fits within 128
bits (plus the 1-bit out-of-band tag).
Bounds compression takes advantage of redundancy between the address
and the bounds, which occurs because a pointer typically falls within (or
close to) its associated allocation, and because allocations are typically
well aligned.
The compression scheme uses a floating-point representation, allowing high-precision bounds for small
objects, but requiring stronger alignment and padding for larger allocations.

\subsection{Architectural rules for capability use}

The architecture enforces several important security properties on changes to
this metadata:

\begin{description}
\item[Provenance validity] ensures that capabilities can be used -- for
  load, store, instruction fetch, etc., -- only if they are derived via valid
  transformations of valid capabilities.
  This property holds for capabilities in both registers and memory.

% \item[Capability integrity] prevents direct in-memory manipulation of
%   capabilities.  (Although this property is subsumed
%   under the previous property, it seems worth stating on its own.)
% \pgnnote{Does that added sentence work?}
% \rwnote{I'm not really sure that that helps.}

%  \psnote{As they are stated above, ``provenance validity'' 
%    subsumes ``capability integrity'', which is a bit confusing.  One
%    could just lose the latter, or (I suppose) split ``provenance
%    validity'' into the case of capability construction in registers,
%    via loads and register
%    operations and the case of capability (de)construction in memory,
%    via good and bad store operations}

%  \psnote{As stated above, ``provenance validity'' involves both the
%    construction and use of capabilities. Think that's ok, but a
%    different slicing of the concepts would be to have it just be
%    construction.   If sticking with ``can be used'', then somehow the
%    text should be elaborated to not forbid use in non-authorising
%    ways, e.g., ``using'' a possibly-non-valid capability by pulling
%    out its address}
  
\item[Monotonicity] requires that any capability derived from another
  cannot exceed the permissions and bounds of the capability from which it was derived.

% \psnote{That's a bit odd, as a capability is really just a pure
%   value, not a mutable thing.  A better (but still a bit fuzzy)
%   statement would be something like ``Monotonicity ensures that a
%   valid capability can only be constructed from another capability
%   with the same or greater authority''.  
% Beyond that, one should talk not just about monotonicity of capability
% construction, but monotonicity of the set of all \emph{reachable
%   capabilities} -- compare with the cheri\_formal\_paper Section IV. }
% \rwnote{This comment partially addressed.}
% \rwnote{This report does not discuss compartmentalisation at all, so is
% uninterested in the other definition of monotonicity.}
  
\end{description}

At boot time, the architecture provides initial capabilities to the firmware,
allowing data access and instruction fetch across the full address space.
Additionally, all tags are cleared in memory.
Further capabilities can then be derived (in accordance with the monotonicity
property) as they are passed from firmware to boot loader, from boot loader to
hypervisor, from hypervisor to the OS, and from the OS to the application.
At each stage in the derivation chain, bounds and permissions may be
restricted to further limit access.
For example, the OS may assign capabilities for only a limited portion of the
address space to the user software, preventing use of other portions of the
address space.

% These capabilities describe the set of memory access permissions held by each
% software component.

%The initial capabilities are then
%  derived from existing valid capabilities, in accordance with the monotonicity
%  property;
% \pgnnote{This seems cleaner.}
% \rwnote{More clear but less correct -- the initial capabilities are never
%   derived from other capabilities.}

Similarly, capabilities carry with them \textit{intentionality}: when a
process passes a capability as an argument to a system call, the OS kernel can
carefully use only that capability to ensure that it does not access other
process memory that was not intended by the user process -- even though the
kernel may in fact have permission to access the entire address space through
other capabilities it holds.
This is important as it prevents ``confused deputy'' problems, in which a more
privileged party uses an excess of privilege when acting on behalf of a less
privileged party, performing operations that were not intended to be
authorized.
For example, this prevents the kernel from overflowing the bounds on a
userspace buffer when a pointer to the buffer is passed as a
system-call argument.

These architectural properties provide the foundation on which a
capability-based OS, compiler, and runtime can implement C/C++-language memory
safety.

\subsection{Pure-capability CHERI C/C++}

The architectural-capability type can be used in a variety of ways by
software.
One particularly useful use case is in implementing \textit{pure-capability
C/C++}.
In this model, all C/C++ language-visible pointer types, as well as any
implied pointers implementing vtables, return addresses, global variables,
arrays of variadic-function arguments, and so on, are implemented using
capabilities with tight bounds.
This allows the architecture to imbue pointers with protection by virtue of
architectural provenance validity, bounds checking, and permission checking,
protecting pointers from corruption and providing strong spatial memory
safety.

Pure-capability code executes within a capability-aware run-time environment
-- whether ``bare metal'' with a suitable runtime, or in a richer, OS-based
process environment such as CheriABI (see Section~\ref{sec:cheriabi}),
which ensures that:
\begin{itemize}
  \item capabilities are context switched (if required);
  \item tags are maintained by the OS virtual-memory subsystem (if present);
  \item capabilities are supported in various OS control operations such as
    debugging (as needed);
  \item system-call arguments, the
run-time linker, and other aspects of the OS Application Binary Interface
(ABI) utilize capabilities rather than integer pointers; and
  \item the C/C++-language runtime implements suitable capability preservation
    \\
    (e.g., in \cfunc{memcpy}) and restriction (e.g., in \cfunc{malloc}).
\end{itemize}
CheriABI operates as a complete additional OS ABI, much in the style of a
32-bit or 64-bit OS personality, in that it requires its own set of suitably
compiled system libraries and classes.
We have also successfully adapted bare-metal runtimes, such as newlib, and
embedded operating systems, such as RTEMS, to support CHERI memory protection.

Outside of the OS and language runtime themselves, this compilation mode
requires relatively few source-code-level changes to C/C++-language software,
which are explored in the remainder of this document.

\subsection{Referential, Spatial, and Temporal Safety}

CHERI pure-capability C/C++ introduces a number of new types of protection not
present in compilation to conventional architectures:

\begin{description}
\item[Referential Safety] protects pointers (references) themselves.
  This includes \textit{integrity} (corrupted pointers cannot be dereferenced)
  and \textit{provenance validity} (only pointers derived from valid pointers
  via valid manipulations can be dereferenced).

  When pointers are implemented using architectural capabilities, CHERI's
  capability tags and provenance validity naturally provide this protection.

\item[Spatial Safety] Prevents manipulation of a pointer outside of the bounds
  of its associated allocation from granting access to another allocation.

  This is accomplished by various memory allocators, including the run-time
  linker for global variables, the stack allocator, and the heap allocator,
  settings the bounds on the capability implementing pointer before returning
  it to the caller.
  Due to precision constraints on capability bounds, bounds on returned
  pointers may include additional padding, but will not permit access to any
  other allocations.
  Monotonicity ensures that callers cannot later broaden the bounds to cover
  other allocations.

\item[Temporal Safety] Prevents a pointer retained after a call to
  \cfunc{free} from granting access to future \cfunc{malloc} memory allocators
  via that pointer.

  This is accomplished by preventing new pointers being returned to a
  previously allocated region of memory while any pointers to the prior
  allocation persist in application-accessible memory.
  Memory will be held in \textit{quarantine} until any such pointers have been
  atomically revoked; then the memory may be reallocated.
  Architectural capability tags allow pointers implementing using capabilities
  to be accurately and efficiently located and overwritten in memory.
  Spatial safety ensures that pointer cannot be used to reference other
  memory, including freed memory.
\end{description}

\textbf{Referential safety} and \textbf{spatial safety} are implemented by our
pure-capability bare-metal, CheriFreeRTOS, CHERI-RTEMS, and CheriBSD's CheriABI
execution environments.
\textbf{Temporal safety} is currently an experimental feature present only in
CheriBSD, and is not yet on the main development branch; we have no plans to
develop support for temporal memory safety in CheriFreeRTOS and CheriRTEMS.

\section{Impact on the C/C++ programming model}

Several types of changes may be required by programmers; the extent to which
these changes impact a particular library or application will depend
significantly on its idiomatic use of C.
Our experience suggests that low-level system components such as run-time
linkers, debuggers, memory allocators, and language runtimes require a modest
but non-trivial porting effort.
Similarly, support classes that include, for example, custom synchronization
features, may also require moderate adaptation as well.
Other applications may compile with few or no changes -- especially if they
are already portable across 32-bit and 64-bit platforms and are written in a contemporary C or C++ dialect.
In the following sections, we consider various types of programmer-visible
changes required in the pure-capability programming environment.

\subsection{CHERI-related header files}

Two new header files provide access to CHERI-related programming interfaces:

\begin{description}
\item[\pathname{cheri/cheri.h}] defines constants such as those used in the
  capability permission mask.

\item[\pathname{cheri/cheric.h}] defines interfaces to access and
  modify capability properties.
\end{description}

\subsection{Capability-related faults}
\label{sec:faults}

When capability properties are violated, such as an attempt to dereference an
invalid capability, access memory outside the bounds of a capability, or perform
accesses not authorized by the permissions on a capability, this typically
leads to a hardware exception (trap).
In the CheriABI process environment, the operating system catches the hardware
exception and delivers a \SIGPROT signal to the user process;
further information may be found in Section~\ref{sec:cheriabi}.

\rwnote{We've opted to use the term ``hardware exception'' throughout, and
  mention ``traps'' only here.  This could cause confusion with respect to C++
  exceptions .. but perhaps less so than if we used the word ``exception''
  unadorned.}

\subsection{Pointer provenance validity}
\label{sec:pointer_provenance_validity}

Pure-capability CHERI C and C++ implement pointers using architectural
capabilities, rather than using conventional 32-bit or 64-bit integers.
This allows the provenance validity of language-level pointers to be
protected by the provenance properties of CHERI architectural capabilities:
only pointers implemented using valid capabilities can be dereferenced.
Other types that contain pointers, such as \cintptrt, are similarly implemented
using architectural capabilities, so that casts through these types
can retain capability properties.
When a dereference is attempted on a capability without a valid tag --
including load, store, and instruction fetch -- a hardware exception fires
(see Section~\ref{sec:faults}).

On the whole, the effects of pointer provenance validity are non-disruptive to
C/C++ source code.
However, a number of cases exist in language runtimes and other
(typically less portable) C code that conflate integers and pointers that can
disrupt provenance validity.
In general, generated code will propagate provenance validity in only two
situations:

\begin{description}
\item[Pointer types] The compiler will generate suitable code to propagate
  the provenance validity of pointers by using capability load and store
  instructions.
  This occurs when a pointer type is used (e.g., \cvoidstar), or when an
  integer type defined as being able to hold a pointer is used (e.g.,
  \cintptrt).
  As with attempting to store 64-bit pointers in 32-bit integers on 64-bit
  architectures, passing a pointer through an inappropriate type will lead to
  truncation of metadata (e.g., the validity tag and bounds).
  It is therefore important that a suitable type be used to hold pointers.

  This pattern often occurs where an opaque field exists in a data structure
  -- e.g., a \clongt argument to a callback in older C code -- that
  needs to be changed to use a capability-oblivious type such as \cintptrt.

\item[Capability-oblivious code] In some portions of the C/C++ runtime and
  com\-piler-generated code, it may not be possible to know whether memory is
  intended to contain a pointer or not -- and yet preserving pointers is
  desirable.
  In those cases, memory accesses must be performed in a way that preserves
  pointer provenance.
  In the C runtime itself, this include \cfunc{memcpy}, which must use
  capability load and store instructions to transparently propagate capability
  metadata and tags.

  A useful example of potentially surprising code requiring modification for
  pure-capability CHERI C is \cfunc{qsort}.
  Some C programs assume that \cfunc{qsort} on an array of data structures
  containing pointers will preserve the usability of those pointers.
  As a result, \cfunc{qsort} must be modified to perform memory copies using
  pointer-based types, such as \cintptrt, when size and alignment
  require it.
\end{description}

\subsubsection{Recommended use of C-language types}

As confusion frequently arises about the most appropriate types to use for
integers, pointers, and pointer-related values, we make the following
recommendations:

\begin{description}
\item[\cintt, \cintttt, \clongt, \cintsft,
  \ldots{}] These pure integer types should be used to hold integer values
  that will never be cast to a pointer type without first combining them with
  another pointer value -- e.g., by using them as an array offset.
  Most integers in a C/C++-language program will be of these types.

\item[\vaddrt] This integer type should be used to hold virtual
  addresses.
  \vaddrt should not be directly cast to a pointer type for
  dereference; instead, it must be combined with an existing valid capability
  to the address space to generate a dereferenceable pointer.
  Typically, this is done using the \ccode{cheri\_setaddress(c, x)} function.

\item[\sizet, \ssizet] These integer types should be used
  to hold the unsigned or signed lengths of regions of virtual address space.
  \arnote{\size not necessary the same as unsigned \ptrdifft.}

\item[\ptrdifft] This integer type describes the integer distance
  between the addresses of two pointers, and should not be used for
any other purpose.
  It can be added to a pointer to obtain a new pointer, but the result will
  be dereferenceable only if the address lies within the bounds of the
  pointer from which it was derived.

  \note{Isn't that last sentence true of any combination?}{nwf}

  Less standards-compliant code sometimes uses \ptrdifft when the
  programmer more likely meant \cintptrt or (less commonly)
  \sizet.
  When porting code, it is worthwhile to audit use of \ptrdifft.

  \note{Should we recommend that \size be used to hold lengths of
  allocations and \ptrdifft be used to talk about spans of
  address space (e.g., the offsets between two subobjects of an allocation)?  I feel
  like the recommendations here are not as concrete as I'd like.}{nwf}

\item[\cintptrt, \cuintptrt] These integer types should be
  used to hold values that may be valid pointers if cast back to a pointer
  type.
  When an \cintptrt is assigned an integer value -- e.g., due to
  constant initialization to an integer in the source -- and the result is
  cast to a pointer type, the pointer will be invalid and hence
  non-dereferenceable.
  These types will be used in two cases: (1) Where there is uncertainty as to
  whether the value to be held will be an integer or a pointer -- e.g., for an
  opaque argument to a callback function; or (2) Where it is more convenient
  to place a pointer value in an integer type for the purposes of arithmetic.
  
  \note{The observable, integer range of a \uintptr is the same as
  that of a \vaddrt, despite the increased \emph{alignment} (and,
  ``incidentally,'' \emph{storage}) requirements, yes?}{nwf}

\item[\ccharstar, \ldots{}] These pointer types are suitable for
  dereference, but in general should not be cast to or from arbitrary integer
  values.
  Valid pointers are always derived from other valid pointers, and cannot be
  constructed using arbitrary integer arithmetic.
\end{description}

It is important to note that \cuintptrt is no longer the same size as
\sizet. This difference may require making some changes to
existing code to use the correct type depending on whether the variable
needs to be able store a pointer type. In cases where this is not obvious
(such as for a callback argument), we recommend the use of \cuintptrt.
While this does double the size of the variable and can result in slightly
less efficient code generation, it ensures that provenance is maintained.

\pgnnote{The above section begs questions relating to what is the
  responsibility of programmers and what can be aided or managed by
  compilers.  Ideally, the latter would be preferable to requiring
  programmers to understand things are are possibly beyond there so-called
  experience.}

\subsubsection{Capability alignment in memory}

Because tags apply only to capability-aligned, capability-sized locations in
memory, unaligned storage of pointers will either generate a run-time
hardware exception (if a capability-aware load or store is performed), or discard the
tag (if a capability-oblivious memory copy is performed -- e.g., using
\cfunc{memcpy} to copy from an aligned location to an unaligned one).
One example of this is Berkeley DB (BDB) when used as an in-memory
implementation rather than as an on-disk database format.
Even when patched to use \cfunc{memcpy} to copy objects stored as data, it
does not ensure sufficient alignment in its internal storage to preserve tags.
We therefore recommend against using BDB for this purpose.
While unaligned pointer use is uncommon in C programs, as data-structure
layouts are normally designed to keep them strongly aligned for performance
and atomicity reasons reasons, code depending on unaligned pointers will need
to be changed.

\amnote{Should we mention code that assumes that it is ok to go out of bounds
for optimization purposes? E.g., strcmp loading a word at a time?}

\subsubsection{Single-origin provenance}

\textbf{XXXRW: This section will get rewritten soon, as we have made changes
  here, such as using static analysis to identify the most appropriate
  provenance when combining a pointer and an integer constant or cast
  non-provenance bearing type, and in the area of warnings.}

In the CHERI architecture, capabilities are derived from a single other
capability.
However, in C code, expressions may construct a new \cintptrt value from more
than one provenance-carrying parent \cintptrt{} -- for example, by casting both a
pointer and a literal value to \cintptrt{}'s, and then adding them.
In that case, the compiler must decide which input capability provides the
capability metadata (bounds, permissions, \ldots{}) to be used in the output
value.

We currently specify that, in CHERI C/C++, the compiler will select the
left-hand-most pointer in the expression.
However, this can lead to surprising behavior if an expression places the
preferred pointer to the right-hand-side of another argument that will not
have appropriate provenance:

\begin{lstlisting}[language=C]
void *c2 = (void *)(1 + (uintptr_t)c1);
\end{lstlisting}

In C with integer pointers, this \cvar{c2} might be expected to have the
same value as \cvar{c1}, except that the address had been incremented by one.
However, in CHERI C, the constant \cconst{1} is promoted to a capability
type, \cuintptrt, which has invalid pointer provenance; because it
is on the left-hand side, its tag bit, bounds, and permissions will be used
rather than those associated with \cvar{c1}, leading to \cvar{c2} being
non-dereferenceable.
The nearly identical expression swapping the order of those arguments in the
expression will give the desired value, in which \cvar{c1}'s tag, bounds,
and permissions are used:

\begin{lstlisting}[language=C]
void *c2 = (void *)((uintptr_t)c1 + 1);
\end{lstlisting}

\textbf{NOTE:
We are currently exploring alternative semantics -- such as selecting
pointer-originated provenance in preference to integer-originated provenance,
and causing a compilation error in the event that the result is ambiguous.
We hope that this will reduce potential code disruption and programmer
confusion arising from single-origin provenance for \cintptrt.}

\subsection{Bounds}

CHERI C/C++ pointers are implemented using capabilities that enforce lower and
upper bounds on access.
In the pure-capability run-time environment, those bounds are normally set to
the range of the memory allocation into which the pointer is intended to
point.
Because of capability compression, increased alignment requirements may apply
to larger allocations (see Section~\ref{sec:bounds_alignment}).

Bounds may be set on pointers returned by multiple system components including
the OS kernel, the run-time linker, compiler-generated code, system libraries,
and other utility functions.
As with violations of provenance validity, out-of-bounds accesses -- including
load, store, and instruction fetch -- trigger a hardware exception (see
Section~\ref{sec:faults}).

\subsubsection{Bounds from the compiler and linker}

The compiler will arrange that pointers to stack allocations have suitable
bounds, and that the run-time linker will return bounded pointers to global
variables.
Bounds will typically be set based on an explicitly requested allocation size
(e.g., via the size passed to \cfunc{alloca}) or, for compiler-generated
code or linker-allocated memory, by the C type mechanism (e.g.,
\ccode{sizeof(foo)}), adjusted for precision requirements arising from
capability compression.
In some cases, such as with global variables allocated in multiple object
files, the actual size of the allocation may not be resolved until run time,
by the run-time linker.
These bounds will typically not cause unexpected behavior.

\subsubsection{Bounds from the heap allocator}

\cfunc{malloc} will set bounds on pointers to new heap allocations.
In typical C use, this is not a problem, as programmers expect to access
addresses only within an allocation.

However, in some uses of C, there may be an expectation that memory access can
occur outside of the allocation bounds of the pointer via which memory access
takes place.
For example, if an integer pointer difference \cvar{D} is taken between
pointers to two different allocations (\cvar{B} and \cvar{A}), and later
added to pointer \cvar{A}, the new pointer will have an address
within \cvar{B}, but permit access only to \cvar{A}.
% This idiom is not frequently used, but is particularly prevalent within
% memory allocators, linkers, and language runtimes.
% \bdnote{I don't think the list of things is representative.  The one place I can recall off hand is a tcsh bug.}
This idiom is mostly likely to be found with non-trivial uses of \cfunc{realloc} (e.g., cases where multiple pointers into a buffer allocated or reallocated by \cfunc{realloc} need to be updated).
We note that the subtraction of two pointers from different
allocations is undefined behaviour in ISO C, and risks mis-optimization from
breaking compiler alias analysis assumptions.

\subsubsection{Subobject bounds}

Pure-capability CHERI C and C++ also support automatically restricting the
bounds when a pointer is taken to a subobject -- for example, an array
embedded within another structure that itself has been heap allocated.
This will prevent an overflow on that array from affecting the remainder of
the structure, improving spatial safety.

However, subobject bounds also prevent use of the \ccode{containerof}
pattern, in which pointer arithmetic on a pointer to a subobject is used to
recover a pointer to the container object -- for example, as seen in the
widely used BSD \pathname{queue.h} linked-list macros or the generic C
hash-table implementation, \pathname{uthash.h}.

In these cases, an opt-out annotation can be applied to a given type, field or variable
that instructs the compiler to not tighten bounds when creating pointers to subobjects.
We currently define three opt-out annotations that can be used to allow
existing code to disable use of subobject bounds:

\paragraph{Completely disable subobject bounds} It is possible to annotate a typedef, record member,
or variable declaration with:

\begin{lstlisting}[language={C}]
__attribute__((cheri_no_subobject_bounds))
\end{lstlisting}

\noindent
to indicate that the compiler should not tighten bounds when taking the address or a C++ reference. In C++11/C20 mode this can also be spelled as \cxxcode{[[cheri::no\_subobject\_bounds]]}.

\begin{lstlisting}[language={C}]
struct str {
    /*
     * Nul-terminated string array -- pointers taken to this subobject will
     * use the array's bounds, not those of the container structure.
     */
    char               str_array[128];

    /*
     * Linked-list entry element -- because of the additional attribute,
     * pointers taken to this subobject will use the container structure's
     * bounds, not those of the specific field.
     */
    struct list_entry  str_le __attribute__((cheri_no_subobject_bounds));
} str_instance;

void
fn(void)
{
    /* Struct pointer gets bounds of str_instance. */
    struct str *strp = &str_instance;

    /* Character pointer gets bounds of the subobject, not str_instance. */
    char *c = str_instance.le_array;

    /* Struct pointer gets bounds of str_instance, not the subobject. */
    struct list_entry *lep = &str_instance.str_le;
}
\end{lstlisting}

\paragraph{Disable subobject bounds in specific expressions}
It is also possible to opt out of bounds-tightening on a per-expression
granularity by casting to an annotated type:

\begin{lstlisting}[language={C}]
char *foo(struct str *strp) {
    return (&((__attribute__((cheri_no_subobject_bounds))struct str *)
        strp)->str_array);
}
\end{lstlisting}

\paragraph{Use remaining allocation size}
In certain cases, the size of the subobject is not known, but we still know that data
before the field member will not be accessed (e.g., variable size array members
inside structs).
Pre-C99 code will declare such members as fixed-size arrays which will cause
a hardware exception if the allocation does not grant access to that many bytes.\footnote{%
If flexible arrays members are declared using the C99 syntax with empty square
brackets the compiler will automatically use the remaining allocation size.}
To use the remaining allocation size instead of completely disabling bounds
(and thus protecting against buffer underflows) the annotation:

\begin{lstlisting}[language={C}]
__attribute__((cheri_subobject_bounds_use_remaining_size))
\end{lstlisting}

\noindent
can be used.
When targeting C++11/C20:

\begin{lstlisting}[language={C++}]
[[cheri::subobject_bounds_use_remaining_size]]
\end{lstlisting}

\noindent
is also supported.
Examples of this pattern include FreeBSD's \ccode{struct dirent} which uses
\ccode{char d\_name[255]} for an array that is actually of variable size, with
the containing allocation (e.g., of the heap) being sized to allow additional
space for array entries regardless of size in the type definition.
For example:

\begin{lstlisting}[language={C}]
struct message {
    int     m_type;

    /*
     * Variable-length character array -- because of the additional
     * attribute, pointers taken to this subobject will have a lower bound
     * at the first address of the array, but retain an upper bound of the
     * allocation containing the array, rather than 252 bytes higher.
     */
    char    m_data[252]
                 __attribute__((cheri_subobject_bounds_use_remaining_size));
};
\end{lstlisting}

\textbf{NOTE:
The use of subobject bounds imposes additional compatibility constraints on
existing C and C++ code. While we have not encountered many issues related to
subobject bounds in existing code, it does slightly increase the porting effort.
Therefore, this feature is currently not enabled by default and requires a compiler
flag to be enabled; we hope to enable it by default in the future.}

% \noindent
% \textbf{XXX: Explain how to exempt those pointer-taking snippets.}

\subsubsection{Other sources of bounds}

Bounds may also be set by other parts of the implementation.
For example, the kernel may set bounds on pointers to new memory mappings (see
Section~\ref{sec:cheriabi}), and the system library may set bounds on pointers
into returned buffers from APIs -- e.g., \cfunc{fgetln}.
More detailed information on how C/C++ code can set bounds can be found in
Section~\ref{sec:setbounds}.

\subsubsection{Out-of-bounds pointers}

The capability compression model exploits redundancy between the pointer and
its bounds to reduce memory overhead.
However, when a pointer goes out of bounds, this redundancy is reduced, and at
some point the bounds can no longer be represented within the capability.
As a result, the architecture prohibits manipulations which would produce such
a capability; in general, attempts to do so will clear the tag, resulting in
an invalid capability -- leading attempting dereferences to fail in the same
manner as a loss of pointer provenance validity (see
Section~\ref{sec:pointer_provenance_validity}).
\psnote{Comment on whether that should immediately trap instead?}
ISO C permits pointers to go only one byte beyond their original
allocation, but real code often \psnote{not sure how strong to be there}
constructs transient pointer values outside that. 
CHERI permits the address of a capability to go further out, with
%of bounds than the one byte strictly required by ISO C;
the exact distance permitted being 
a function of the length of the bounded region:

\begin{itemize}
\item A pointer is able to travel at least 1/4 the size of the object, or
  2KiB, whichever is greater, above its upper bound.

\item It is able to travel at least 1/8 the size of the object, or 1KiB,
  whichever is greater, below its lower bound.
\end{itemize}

In general, programmers should not rely on support for arbitrary out-of-bounds
pointers.  Nevertheless, in practice, we have found that the CHERI capability
compression scheme supports almost all in-the-field out-of-bounds behavior in
widely used software.  

\subsection{Pointer comparison}

In pure-capability CHERI C and C++, pointer comparison considers only the
integer address portions of a capability.
This means that differences in tag validity, bounds, permissions, and so on,
will not be considered when by C operators such as $==$, $<$, and $<=$.
On the whole, this leads to intuitive behavior in systems software, where,
for example, \cfunc{malloc} adjust bounds on a pointer before returning it to
a caller, and then expect an address-wise comparison to succeed when the
pointer is later returned via a call to \cfunc{free}.
However, this behavior could also lead to potentially confusing results; for
example:

\begin{itemize}
\item If a tag on a pointer is lost due to non-provenance-preserving
  \cfunc{memcpy}(e.g., a \ccode{for} loop copying a sequence of bytes), the
  source and destination pointers will compare as equal even though the
  destination will not be dereferenceable.

\item If a \cfunc{realloc} implementation returns a pointer to the same
  address, but with different bounds, a caller check to see if the passed and
  returned pointers are equal will return \ccode{true} even though an access
  might be permitted via one pointer but not the other.
\end{itemize}

\noindent
Practical experience has suggested that the current semantics produce fewer
subtle bugs, and require fewer changes, than having comparison operators take
the tag or other metadata into account.

\subsection{Bitwise operations on capability types}

In most cases bitwise operations -- such as those used to store or clear flags
in the lower bits of pointers to well-aligned allocations -- will result in the expected \cuintptrt value being created.
However, there are some corner cases where the result may be a tagged (but out-of-bounds)
capability when an integer value is expected.
\arnote{TODO: add an example. Maybe the mutex example checking low pointer bits + some alignment checks?}
This may also result in the loss of tags if intermediate results become unrepresentable.\footnote{%
Previous versions of the compiler used the capability offset (address minus base) instead
of the address for arithmetic on \cuintptrt.
This often resulted in unexpected results and therefore we switched to using
the address in \cuintptrt arithmetic instead.
The old offset-based mode may be interesting for garbage collected C where
addresses are less useful and therefore it can still be enabled by
passing \commandline{-cheri-uintcap=offset}.
However, this may result in significantly reduced compatibility with legacy C code.
}
As this can result in hard-to-debug failure modes, the compiler will emit a
warning when using bitwise-AND on capability types such as \cuintptrt.
Most bitwise operations on \cuintptrt fall into one of three categories for which we provide
higher-level abstractions.

\paragraph{Aligning pointer values}
If the C code is attempting to align a pointer or check the alignment of pointers,
the following compiler builtins should be used instead:
\begin{description}
\item[\ccode{\_Bool \_\_builtin\_is\_aligned(T ptr, size\_t alignment)}]
  This builtin returns \cconst{true} if \cvar{ptr} is aligned to at least \cvar{alignment} bytes.
\item[\ccode{T \_\_builtin\_align\_down(T ptr, size\_t alignment)}]
  This builtin returns \cvar{ptr} rounded down to the next multiple of \cvar{alignment}.
\item[\ccode{T \_\_builtin\_align\_up(T ptr, size\_t alignment)}]
  This builtin returns \cvar{ptr} rounded up to the next multiple of \cvar{alignment}.
\end{description}

\rwnote{It would be nice if we had, and could document here, cheri\_ versions
  of these macros.}

One advantage of these builtins compared to \cuintptrt arithmetic is that they preserve the
type of the argument and can therefore remove the need for intermediate casts to \cuintptrt.

\paragraph{Storing additional data in pointers}
\label{sec:low-pointer-bits}
In many cases the minimum alignment of pointer values is known and therefore
programmers assume that the low bits (which will always be zero) can be
used to store additional data.\footnote{%
CHERI actually provides many more usable bits than a conventional architecture:
In the current implementation of 128-bit CHERI, any bit between the least
significant and the 9th least significant bit may be toggled without causing
the tag to be cleared in pointers that point to the beginning of an allocation.
If the pointer is strongly aligned, it may be possible to store even more additional bits.}
The compiler-provided header \ccode{<cheri.h>} provides explicit macros for this
use of bitwise arithmetic on pointers:

\begin{description}
\item[\ccode{uintptr\_t cheri\_set\_low\_ptr\_bits(uintptr\_t ptr, vaddr\_t bits)}]
  This builtin returns \cvar{ptr} with the low bits set to \cconst{bits}. In order to retain
  compatibility with a non-CHERI architecture, \cconst{bits} should be less than the
  known alignment of \cvar{ptr}.

\item[\ccode{vaddr\_t cheri\_get\_low\_ptr\_bits(uintptr\_t ptr, vaddr\_t mask)}]
  This builtin returns the low bits of \cvar{ptr} in the same way as \ccode{ptr \& mask}.
  It should be used instead of the raw bitwise operation since it can never return
  an unexpectedly tagged value.
  \cvar{mask} should be a bitwise-AND mask less than \ccode{\_Alignof(ptr) - 1}.

\item[\ccode{uintptr\_t cheri\_clear\_low\_ptr\_bits(uintptr\_t ptr, vaddr\_t mask)}]
  This builtin clears the low bits of \cvar{ptr} in the same way as \ccode{ptr \& \~mask}.
  It returns a new \cuintptrt value that can be used for memory accesses when cast to a pointer.
  \cvar{mask} should be a bitwise-AND mask less than \ccode{\_Alignof(ptr) - 1}.
\end{description}

\paragraph{Computing hash values}

The compiler will also warn when operators such as modulus or shifts are used on
\cuintptrt. This usually indicates that the pointer is being used as the input to a hash
function or similar computations.
In this case, the programmer should not be using \cuintptrt but instead cast the pointer
to \vaddrt and perform the arithmetic on this type instead.
This has the advantage that it will be slightly more efficient than \cuintptrt arithmetic on
a split-register file architecture such as CHERI-MIPS.


\subsection{Function prototypes and calling conventions}

CHERI C/C++ distinguishes between integer and pointer types at an
architectural level, which can lead to compatibility problems with older C
programming styles that fail to unambiguously differentiate these types:

\begin{description}
\item[Unprototyped (\textit{K\&R}) functions] Because pointers can no longer
  be loaded and stored without using capability-aware instructions, the
  compiler must know whenever a load or store might operate on a pointer
  value.
  The C-language default of using an integer type for function arguments when
  there is not an appropriate function prototype will cause pointer values to
  be handled improperly; this is also true on LP64 ABIs (e.g., most 64-bit
  POSIX systems).
  The compiler will warn when a function without a declared prototype is
  called.\footnote{If the \textit{K\&R} function is defined within the same
  file, the compiler can determine the correct calling convention and will not
  emit a warning.}
  This warning is less strict than \commandline{-Wstrict-prototypes} and can be
  used to convert \textit{K\&R} functions that may cause problems.
  This should not be an issue for C code written in the last 20 years, but
  many core operating-system components can be significantly older.

\item[Variadic arguments] The calling convention for variadic functions
  passes all variadic arguments via the stack and accesses them via an
  appropriately bounded capability.
  This provides memory-protection benefits, but means that vararg functions
  must be declared and called via a correct prototype.

  Some C code assumes that the calling convention of variadic and non-variadic
  functions is sufficiently similar that they may be used interchangeably.
  Historically, this included the FreeBSD kernel's implementation of
  \cfunc{open}, \cfunc{fcntl}, and \cfunc{syscall}.

  \rwnote{I wonder if we need to be more specific with an example here.}

\end{description}

\subsection{Data-structure and memory-allocation alignment}

CHERI C/C++ have stronger alignment requirements than C/C++ on conventional
architectures.
These requirements arise from two sources: that capabilities themselves must
be aligned at twice the integer architectural pointer width, and that
capability compression constraints the addresses that can be used for bounds
on larger objects.
\amnote{Is is worth mentioning compiler flags to warn on excessive padding?
  In particular it seems that it is often the case that the ordering of
  struct elements that was devised for 32bit and 64bit architectures does
  not help much to avoid extra padding with capabilities. It more or less
  depends on how much the pointers are scattered in the struct definition.}

\subsubsection{Restrictions in capability locations in memory}

CHERI C/C++ constrain how and where pointers can be stored in memory in two
ways:

\begin{description}
\item[Alignment] CHERI's tags are associated with capability-aligned,
  capability-sized locations in physical memory.
  Because of this, all valid pointers must be stored at such locations,
  potentially disrupting code that may use other alignments.

  On the whole, for performance and atomicity reasons, pointers are strongly
  aligned even on non-tagged architectures -- however, when C constructs such
  as \ccode{\_\_packed} are used, unaligned pointers can arise, and will not
  work with CHERI.
  While the compiler and native allocators (stack, heap, \ldots{}) will
  provide sufficient alignment for capability-based pointers, custom
  allocators may align allocations to \ccode{sizeof(intmax\_t)} rather than
  \ccode{sizeof(intptr\_t)}.

\item[Size] CHERI capabilities are twice the size of an integer able to
  describe the full address space.
  On 64-bit systems, this means that CHERI pointers will have a width of 128
  bits -- while maintaining the arithmetic properties of a 64-bit integer
  address.
  C code historically embeds assumptions about pointer size in a number of forms,
  all of which will need to be addressed when porting to CHERI,
  including:

  \begin{itemize}
  \item Assuming that a pointer will fit into the largest integer type.
  \item Assuming that the number of bits in a pointer type is the same
    as the number of bits in the address space it can refer to.
  \item Assuming that the number of bits in a pointer type is the same as the
    number of bits suitable for use in performing bit-wise manipulations of
    pointer values.
  \item Assuming that pointers must either be 32 or 64 bits.
  \item Assuming that aligning to \ccode{sizeof(double)} is sufficient to store any type.
  \item Assuming that high bits of the pointer address can be used for
  additional metadata. This is not true on CHERI since toggling high bits of a
  pointer can cause it to be so far out of bounds that it is no longer representable
  due to the compression of pointer bounds. However, it is still possible to use
  the low bits for additional metadata (see section \ref{sec:low-pointer-bits}).
  \end{itemize}
  \rwnote{Should there be more things in this list?}
\end{description}

These portability problems will typically be found due to hardware exceptions
thrown on attempted unaligned accesses of capability values
(see Section~\ref{sec:faults}).
However, they can also arise in the form of stripped tag bits, leading to
invalid capabilities that cannot be dereferenced, if, for example, pointer
values are copied into inappropriately aligned allocations.

\section{APIs for restricting capability permissions and bounds}
\label{sec:setbounds}

\rwnote{I wonder if we should talk more about permissions?  Perhaps not in
  this document, in which case possibly we should talk about them less?}
\amnote{If this is intended as a document to guide porting efforts perhaps
  we should only mention them as background info? If this becomes a summary
  of CHERI programming patterns then we probably want a section that talks
  about permissions as well.}

Although most software does not need to directly manage capability properties,
there are some cases when application code needs to further constrain
permissions or limit bounds associated with pointers.
For example, high-performance applications may contain custom memory
allocators and wish to narrow bounds and permissions on returned pointers
to prevent overflows between its own allocations.
CHERI C therefore provides several new APIs to get and set capability
properties given a pointer argument, which are defined in
\pathname{cheri/cheric.h}.
\amnote{We should make it clear that we are referring to headers in the CheriBSD tree.}

\subsection{Retrieving capability properties}

The following APIs allow capability properties to be retrieved from pointers:

\begin{description}
\item[\ccode{vaddr\_t cheri\_getbase(void *c)}] Return the lower bound of capability
  \cvar{c}.
%  This macro wraps the compiler built-in
%  \cfunc{\_\_builtin\_cheri\_base\_get}.

\item[\ccode{size\_t cheri\_getaddress(void *c)}] Return the address of the capability \cvar{c}.
%  This macro wraps the compiler built-in
%  \cfunc{\_\_builtin\_cheri\_address\_get}.

\item[\ccode{size\_t cheri\_getlen(void *c)}] Return the length of the bounds for the capability \cvar{c}.
%  This macro wraps the compiler built-in
%  \cfunc{\_\_builtin\_cheri\_length\_get}.
  (The base plus the length gives the upper bound on \cvar{c}'s address.)

\item[\ccode{size\_t cheri\_getoffset(void *c)}] Return the difference between the address and the lower bound of the capability \cvar{c}.
 %  This macro wraps the compiler built-in
 %  \cfunc{\_\_builtin\_cheri\_offset\_get}.

\item[\ccode{size\_t cheri\_getperm(void *c)}] Return the permissions of capability
  \cvar{c}.
  (See Section~\ref{sec:available_permissions}.)
%  This macro wraps the compiler built-in
%  \cfunc{\_\_builtin\_cheri\_perms\_get}.

\item[\ccode{\_Bool cheri\_gettag(void *c)}] Return whether capability \cvar{c} has its
  validity tag set.
%  This macro wraps the compiler built-in
%  \cfunc{\_\_builtin\_cheri\_tag\_get}.
\end{description}

\subsection{Restricting capability properties}

The following APIs allow capability properties to be refined on pointers:

\begin{description}
\item[\ccode{void *cheri\_andperm(void *c, size\_t x)}] Perform a bitwise-AND of capability
  \cvar{c}'s permissions and the value \cvar{x}, returning the new
  capability.
  (See Section~\ref{sec:available_permissions}.)
%  This macro wraps the compiler built-in
%  \cfunc{\_\_builtin\_cheri\_perms\_and}.

\item[\ccode{void *cheri\_cleartag(void *c)}] Clear the tag on \cvar{c}, returning the
  new capability.

\item[\ccode{void *cheri\_csetbounds(void *c, size\_t x)}] Narrow the bounds of capability
  \cvar{c} so that the lower bound is the current pointer value (which may
  have been increased relative to \cvar{c}'s original lower bound), and its
  upper bound is suitable for a length of \cvar{x}.
%  This macro wraps the compiler built-in
%  \cfunc{\_\_builtin\_cheri\_bounds\_set}

\item[\ccode{void *cheri\_setaddress(void *c, vaddr\_t a)}] Returns a new capability with the same permissions and bounds as \cvar{c} with the address set to \cvar{a}.
This builtin can be useful to re-derive a valid pointer from an address.
%  This macro wraps the compiler built-in
%  \cfunc{\_\_builtin\_cheri\_address\_set}.
\end{description}

% \note{Are the references to the \ccode{\_\_builtin\_} forms useful?  Do we
% want to encourage their use or the \pathname{cheric.h} macros?}{nwf}

\subsection{Available permissions}
\label{sec:available_permissions}

A number of capability permissions are available per use; only those relating
to pure-capability memory protection are enumerated here:

\begin{description}
\item[\cconst{CHERI\_PERM\_EXECUTE}] Authorize instruction fetch via this
   capability.

\item[\cconst{CHERI\_PERM\_LOAD}] Authorize data load via this capability.

\item[\cconst{CHERI\_PERM\_STORE}] Authorize data store via this capability.

\item[\cconst{CHERI\_PERM\_LOAD\_CAP}] Authorize capability load via this
  capability.
  If the permission is not present, the tag on the loaded value
  will be silently cleared.

\item[\cconst{CHERI\_PERM\_STORE\_CAP}] Authorize capability store via this
  capability.
  If the permission is not present, and the tag on the stored capability is
  valid, then a hardware exception will be thrown.
\end{description}

\subsection{Bounds alignment due to compression}
\label{sec:bounds_alignment}

When the length of an object exceeds 4KiB, additional alignment requirements
apply to the lower and upper bounds.
This may require a memory allocator to increase the alignment of an
allocation, or increase padding on an allocation, to prevent bounds from
spanning more than one object.
The alignment required for allocations over 4KiB is $2^{E+3}$ bytes, where
$E$ is determined from the length, $l$, by
$E = 52 - \textrm{CountLeadingZeros}(l[64:13])$.

\subsection{Implications for memory-allocator design}

One use case is high-performance applications that contain custom memory
allocators, and wish to narrow the bounds of returned pointers.
Two types of modifications are typically required:

\begin{description}
\item[Changes to alignment to allow for capabilities and bounds]
  Changes relating to alignment fall into two categories:
  First, those required to allow pointers to be stored within allocations,
  which requires that allocations be aligned to pointer width -- 128 bits.
  Second, further alignment changes will be required to ensure that bounds can
  be represented precisely.
  This requires suitably aligning both the bottom and top bounds to exclude
  any other live allocations, as described in
  Section~\ref{sec:bounds_alignment}.

\item[Reaching allocation metadata on \cfunc{free}]
  It is often the case that allocators utilize the value of the pointer passed
  to their custom \cfunc{free} function to locate corresponding metadata --
  for example, by always placing that metadata immediately before the
  allocation, which would be outside of the allocation's bounds.\footnote{%
  When a custom allocator places metadata at the beginning of the allocation,
  care must be taken that the resulting pointer is still strongly aligned.
  While porting programs to run on CHERI, we found multiple sub-allocators
  that used 8 bytes of metadata after the result from \cfunc{malloc}.
  This causes the resulting pointer to no longer be sufficiently aligned to
  store capabilities without faulting or stripping tag bits.
  \note{Does CHERI ISAv7 still fault in any of these scenarios?}{nwf}
  }
  Therefore, some additional work may be required to derive a pointer to the
  allocation's metadata via another global capability, rather than the one
  that has been passed to \cfunc{free}.
\end{description}

\section{The CheriABI POSIX process environment}
\label{sec:cheriabi}

The CheriABI process environment implements a standard POSIX/UNIX API, but in
some areas there are changes to API semantics (e.g., in the handling of tagged
pointer values and I/O) or new functionality (such as relates to handling
capability-related faults).

\subsection{POSIX API changes}

\begin{description}
\item[Writing and reading pointers via files] In the CheriABI process
  environment, only untagged data, not tagged pointers, may be written to or
  read from files.
  If a region of memory containing valid pointers is written to a file, and
  then read back, the pointers in that region will no longer be valid.
  If a file is memory mapped, then pages mapped copy-on-write
  (\cconst{MAP\_PRIVATE}) are able to hold tagged pointers, since they are in
  fact swap backed rather than file backed, but pages mapped directly from the
  buffer cache (\cconst{MAP\_SHARED}) are not.

\item[Passing pointers via IPC] In the CheriABI process environment, only
  untagged data, not tagged pointers, may be passed via various forms of
  message-passing Inter-Process Communication (IPC).
  Some existing software takes advantage of a shared address-space layout
  (via \cfunc{fork}) to pass pointers to elements of shared data structures
  (e.g., entries in dispatch tables).
  This code must be converted to use indexes into tables or other lookup
  mechanisms rather than passing pointers via IPC.

\item[\cfunc{mmap} bounds] In CheriABI, the \cfunc{mmap} system
   call returns a bounded capability to allocated address space.
   The requested mapping must be rounded up in length to ensure that the
   the returned capability does not overlap with unallocated (or otherwise
   allocated) regions.
   This currently requires code changes in consumers of \cfunc{mmap}.

\item[\cfunc{mmap} permissions] The permissions of the capability
   returned by \cfunc{mmap} are determined by a combination of the
   requested page protections and the capability passed as an address hint
   (or fixed address with \cconst{MAP\_FIXED}).
   When using the pattern of requesting a mapping with \cconst{PROT\_NONE}
   and then filling in sections (as is done in run-time linkers, VM host
   environments, etc), it is necessary to ensure that the initial
   capability has the right permissions.
   The \cvar{prot} argument has been extended to accept additional
   flags indicating the maximum permission the page can have so a
   linker might request a reservation for a library with the permissions
   \ccode{(PROT\_MAX(PROT\_READ|PROT\_WRITE|PROT\_EXEC) | PROT\_NONE)} which
   would return a capability permitting loads, stores, and instruction
   fetch while mapping the pages with no (MMU) permissions.
\end{description}

\subsection{Handling capability-related signals}

When a capability hardware exception fires, the operating system will map it
into the UNIX \SIGPROT signal.
By default, this signal terminates the process, but the signal can be caught
by registering a \SIGPROT handler.
When the signal handler fires, \ccode{siginfo.si\_code} will be set to
describe the cause of the fault; available values include:

\begin{description}
\item[\cconst{PROT\_CHERI\_TAG}] Capability tag fault -- dereferencing an
  invalid capability has been attempted.
\item[\cconst{PROT\_CHERI\_BOUNDS}] Capability bounds fault -- an out-of-bounds
  access has been attempted.
\item[\cconst{PROT\_CHERI\_PERM}] Capability permission fault -- the attempted
  access exceeds the permissions granted by a capability.
\item[\cconst{PROT\_CHERI\_SEALED}] Capability sealed fault -- dereferencing a
  sealed capability has been attempted.
\end{description}

%\section{Potential performance impact}
%
%Pure-capability code performs very similarly to non-capability-based code on
%the same architecture, as most compiler-generated constructions are identical.
%A small number of additional instructions will be used around pointers taken
%to stack allocations, or when making new heap allocations, to set up bounds
%for the returned pointer -- but these typically have negligible cost.
%
%The primary overhead for pure-capability code is therefore the increase in
%pointer size, which can impact data-cache efficiency.
%Performance overhead therefore tends to correspond to the density of pointer
%loads and stores in an application's dynamic memory access pattern.
%Relatively pointer-light programs often have overhead that is sub-1\% -- for
%example, in stream or image processing.
%Programs with more pointer-dense access patterns, such as language runtimes
%and compilers, may see more significant overheads, in the range of 10\%.
%\textbf{XXX: Turn these into more real numbers.}

\section{Further reading}
\label{sec:further_reading}

The primary reference for the CHERI Instruction-Set Architecture (ISA) is the
ISA specification; at the time of writing, the most recent version is CHERI
ISAv7~\cite{UCAM-CL-TR-927}:

\smallskip
\noindent
\url{https://www.cl.cam.ac.uk/techreports/UCAM-CL-TR-927.pdf}
\smallskip

Our technical report, \textit{An Introduction to CHERI}, provides a high-level
overview of the CHERI architecture, ISA modeling, hardware implementations,
and software stack~\cite{UCAM-CL-TR-941}:

\smallskip
\noindent
\url{https://www.cl.cam.ac.uk/techreports/UCAM-CL-TR-941.pdf}
\smallskip

\noindent
We published a paper on CheriABI and the adaptation of a complete OS userspace
and application suite to a pure-capability process environment at ASPLOS
2019~\cite{davis2019:cheriabi}:

\smallskip
\noindent
\url{https://www.cl.cam.ac.uk/research/security/ctsrd/pdfs/201904-asplos-cheriabi.pdf}
\smallskip

\noindent
We have also released an extended technical-report version of this paper that
includes greater implementation detail~\cite{UCAM-CL-TR-932}:

\smallskip
\noindent
\url{https://www.cl.cam.ac.uk/techreports/UCAM-CL-TR-932.pdf}
\smallskip

\noindent
We published a paper on CHERI and temporal memory safety for the heap at
Oakland 2020~\cite{filardo:cornucopia}:

\smallskip
\noindent
\url{https://www.cl.cam.ac.uk/research/security/ctsrd/pdfs/2020oakland-cornucopia.pdf}
\smallskip

\noindent
We published a paper on C-language pointer provenance, and the implications
for software design, at POPL 2019; CHERI C was a case study in the practical
enforcement of capability provenance-validity
enforcement~\cite{cerberus-popl2019}:

\smallskip
\noindent
\url{https://www.cl.cam.ac.uk/research/security/ctsrd/pdfs/201901-popl-cerberus.pdf}
\smallskip

%\textbf{XXX: Any other pointers?}

\section{Acknowledgements}

We gratefully acknowledge the helpful feedback from our colleagues, including
Peter G. Neumann, Nathaniel Wesley Filardo, John Baldwin, Peter Sewell, Brett
Gutstein, Alfredo Mazzinghi, Simon W. Moore, Lee Smith, and Paul Gotch.
This work was supported by the Defense Advanced Research Projects Agency (DARPA) and the Air Force Research Laboratory (AFRL), under contracts
FA8750-10-C-0237 (``CTSRD'') and HR0011-18-C-0016 (``ECATS'').
The views, opinions, and/or findings contained in this report are those of the authors and should not be interpreted as representing the official views or policies of the Department of Defense or the U.S. Government.
We also acknowledge the EPSRC REMS Programme Grant (EP/K008528/1), the
ERC ELVER Advanced Grant (789108), Arm Limited,
HP Enterprise, and Google, Inc.
Approved for Public Release, Distribution Unlimited.

\bibliographystyle{abbrv}
\bibliography{cheri}

\end{document}
